\vspace{1cm}
{ \large \bfseries 4.Συμπεράσματα}\\ % title 4

\begin{justify}
    Αυτή η άσκηση είναι μια πολύ καλή εξάσκηση
    για την κατανόηση της λειτουργίας ενός επεξεργαστή
    ενός κύκλου.Εντρυφήσαμε στην \textlatin{VHDL}, την κατανοήσαμε
    και μάθαμε να χειριζόμαστε τις βιβλιοθήκες της για την
    διευκόλυση της υλοποίησης των κυκλωμάτων. Μέσω εξαντλητικών
    δοκιμών (\textlatin{testbenches}) καταφέραμε να επιβεβαιώσουμε την λειτουργία του επεξεργαστή
    και κατανοήσαμε πόσο σημαντική είναι αυτή η διαδικασία
    για την αποφυγή λαθών και την εξοικονόμηση χρόνου.Τέλος,
    μάθαμε πως να κάνουμε ιεραρχική σχεδίαση, να χωρίζουμε
    τα \textlatin{sub modules} και να τα συνδέουμε σε ένα
    \textlatin{Top Level} αρχείο.
\end{justify}